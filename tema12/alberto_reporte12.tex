% https://es.overleaf.com/latex/templates/project-report/jpzczmpsdzwm

%%% Preamble
\documentclass[paper=leter, fontsize=11pt]{scrartcl}
\usepackage[utf8]{inputenc}
\usepackage[spanish,mexico]{babel}
\usepackage[T1]{fontenc}    % use 8-bit T1 fonts
\usepackage{lmodern}
\usepackage{hyperref}       % hyperlinks
\usepackage{lipsum}
\usepackage[square,numbers]{natbib}
\usepackage{enumitem}

\usepackage[protrusion=true,expansion=true]{microtype}	
\usepackage{amsmath,amsfonts,amsthm} % Math packages
\usepackage[pdftex]{graphicx}
\usepackage{url}
% https://tex.stackexchange.com/a/3785
\usepackage{breqn}
 
\usepackage{booktabs}
\usepackage[table,xcdraw]{xcolor}

\usepackage{tikz}
\usetikzlibrary{positioning,matrix, arrows.meta}

\usepackage{caption} 
\usepackage{subcaption}

\usepackage{multirow}

\usepackage{listings}
\lstdefinestyle{mystyle}{ 
    language=R,
    basicstyle=\ttfamily\footnotesize,
    breakatwhitespace=false,         
    breaklines=true,                 
    captionpos=b,                    
    keepspaces=true,                 
    numbers=left,                    
    numbersep=5pt,                  
    showspaces=false,                
    showstringspaces=false,
    showtabs=false,                  
    tabsize=2
}

\lstset{style=mystyle}
\renewcommand{\lstlistingname}{Código}


\selectlanguage{spanish}
\usepackage[spanish,onelanguage,ruled]{algorithm2e}


%%% Custom sectioning
\usepackage{sectsty}
\allsectionsfont{\centering \normalfont\scshape}


%%% Custom headers/footers (fancyhdr package)
\usepackage{fancyhdr}
\pagestyle{fancyplain}
\fancyhead{}											% No page header
\fancyfoot[L]{}											% Empty 
\fancyfoot[C]{}											% Empty
\fancyfoot[R]{\thepage}									% Pagenumbering
\renewcommand{\headrulewidth}{0pt}			% Remove header underlines
\renewcommand{\footrulewidth}{0pt}				% Remove footer underlines
\setlength{\headheight}{13.6pt}


%%% Equation and float numbering
%\numberwithin{equation}{section}		    % Equationnumbering: section.eq#
%\numberwithin{figure}{section}			    % Figurenumbering: section.fig#
%\numberwithin{table}{section}				% Tablenumbering: section.tab#

%\newtheorem{thm}{Theorem}
%\newtheorem{prop}{Proposition}
%\newtheorem{lemma}{Lemma}
\newtheorem{ex}{Exercise}

%%% Maketitle metadata
\newcommand{\horrule}[1]{\rule{\linewidth}{#1}} 	% Horizontal rule

%%% https://tex.stackexchange.com/a/118217
\usepackage{mathtools}
\DeclarePairedDelimiter\ceil{\lceil}{\rceil}
\DeclarePairedDelimiter\floor{\lfloor}{\rfloor}

\title{
		%\vspace{-1in} 	
		\usefont{OT1}{bch}{b}{n}
		\normalfont \normalsize \textsc{Posgrado de Ingeniería de Sistemas} \\ [25pt]
		\horrule{0.5pt} \\[0.4cm]
		\huge Ejercicios Procesos de ramificación \\
		\horrule{2pt} \\[0.5cm]
}
\author{
		\normalfont 								\normalsize
        Alberto Benavides\\[-3pt]		\normalsize
        \today
}
\date{}


%%% Begin document
\begin{document} 
\maketitle

\begin{ex}[P. 392, e. 1]
  Let $Z_1, Z_2, \ldots, Z_n$ describe a branching process in which each parent has $j$ offspring with probability $p_j$. Find the probability $d$ that the process eventually dies out if
\end{ex}
\begin{enumerate}[label=(\alph*)]
  \item $p_0 = 1 / 2, p_1 = 1 / 4, p_2 = 1 / 4$.
  
  Para este caso, el número esperado de hijos es $m = h'(1) = p_1 + 2p_2 = 1 / 4 + 2 (1 / 4) = 3 / 4 \leq 1$, por lo que por el teorema 10.2, $d = 1$ así que el proceso o herencia o apellido se acabará.
  
  \item $p_0 = 1 / 3, p_1 = 1 / 3, p_2 = 1 / 3$.
  
  Igual que el inciso anterior, $m = 1 / 3 + 2(1 / 3) = 1 \leq 1$, así que $d = 1$.

  \item $p_0 = 1 / 3, p_1 = 0, p_2 = 2 / 3$.
  
  Aquí, $m = 0 + 2 (2 / 3) = 4 / 3 > 1$, pero como $p_0 < p_2$ se puede obtener $d = p_0 / p_2 = \frac{1 / 3}{2 / 3} = 1 / 2$.

  \item $p_j = 1 / 2^{j + 1}$, for $j = 0, 1, 2, \ldots$
  
  En este inciso,
  \begin{dmath*}
    h(z) = p_0 + p_1z + p_2z^2 + p_3z^3 + \ldots
      = 1 / 2^{0 + 1} + 1 / 2^{1 + 1}z + 1 / 2^{2 + 1}z^2 + 1 / 2^{3 + 1}z^3 + \ldots 
      = 1 / 2^{1} + 1 / 2^{2}z + 1 / 2^{3}z^2 + 1 / 2^{4}z^3 + \ldots
      = \frac{1}{2} (1 / 2^{1} + 1 / 2^{2}z + 1 / 2^{3}z^2 + 1 / 2^{4}z^3 + \ldots) / \frac{1}{2}
      = \frac{1}{2} (1 + 1 / 2^{1}z + 1 / 2^{2}z^2 + 1 / 2^{3}z^3 + \ldots)
      = \frac{1}{2} \left(\frac{1}{1 - \frac{1}{2} z}\right)
      = \frac{1}{2 - z}.
  \end{dmath*}

  Si esto es verdad, entonces
  \begin{dmath*}
    h'(z) = \frac{d}{dz} \left( \frac{1}{2 - z} \right)
          = \frac{- \frac{d}{dz} \left( 2 - z \right)}{(2 - z)^2}
          = \frac{1}{(2 - z)^2}
  \end{dmath*}
  por lo que, como $m = h'(1) = \frac{1}{(2 - 1)^2} = 1 \leq 1$, $d = 1$.

  \item $p_j = (1 / 3) (2/3)^j$, for $j = 0, 1, 2, \ldots$
  
  De manera análoga al inciso precedente,
  \begin{dmath*}
    h(z) = p_0 + p_1z + p_2z^2 + p_3z^3 + \ldots
      = \frac{1}{3} \left(\frac{2}{3}\right)^{0} + \frac{1}{3} \left(\frac{2}{3}\right)^1 z^1 + \frac{1}{3} \left(\frac{2}{3}\right)^2 z^2 + \frac{1}{3} \left(\frac{2}{3}\right)^3 z^3 \ldots 
      = \frac{1}{3} \left[1 + \left(\frac{2}{3}\right)^1 z^1 + \left(\frac{2}{3}\right)^2 z^2 + \left(\frac{2}{3}\right)^3 z^3 \ldots\right]
      = \frac{1}{3} \left(\frac{1}{1 - \frac{2}{3} z}\right)
      = \frac{1}{3 - 2z}
  \end{dmath*}
  de donde 
  \begin{dmath*}
    h'(z) = \frac{d}{dz} \left( \frac{1}{3 - 2z} \right)
          = \frac{- \frac{d}{dz} \left( 3 - 2z \right)}{(3 - 2z)^2}
          = \frac{2}{(3 - 2z)^2}
  \end{dmath*}
  por lo que $m = h'(1) = \frac{2}{(3 - 2)^2} = \frac{2}{(1)^2} = 2$ y $d < 1$ cuando $z \neq 1$. Para calcular esta $d$ se obtienen las raíces a partir de igualar $z = h(z)$, así que
  \begin{dmath*}
    z = \frac{1}{3 - 2z}
  \end{dmath*}
  \begin{dmath*}
    2z^2 - 3z + 1 = 0
  \end{dmath*}
  de donde $z_1 = 1$ (ya conocida) y $z_2 = 1/2 = d$.

  \item $p_j = e^{-2} 2^j / j!$, for $j = 0, 1, 2, \ldots$ (estimate $d$ nummerically).
  
  Finalmente, $d = 0.2032$. Esto se obtiene mediante el código \ref{prop2}

\begin{lstlisting}[caption={Aproximación}, captionpos=t, label=prop2]
p = function(j){
  return( exp(-2) * 2 ** j / factorial(j) )
}
d = p(0)
for (m in 1:1000){
  sum = 0
  for (j in 0:100){
    sum = sum + p(j) * (d ** j)
  }
  d = sum
}
d
# 0.2031878699799799
\end{lstlisting}

\end{enumerate}

\begin{ex}[P. 392, e. 3]
  In the chain letter problem (see Example 10.14) find your expected profit if
\end{ex}
\begin{enumerate}[label=(\alph*)]
  \item $p_0 = 1 / 2, p_1 = 0, p_2 = 1 / 2$.
  
  Como $m = p_1 + 2p_2 = 0 + 2(1/2) = 1$, entonces se espera una ganancia de $50 (1 + 1^{12}) - 100 = 0$.
  
  \item $p_0 = 1 / 6, p_1 = 1 / 2, p_2 = 1 / 3$.
  
  Aquí $m = p_1 + 2p_2 = 1/2 + 2(1/3) = 7 / 6$, entonces se espera una ganancia de $50 (7/6 + (7/6)^{12}) - 100 \approx 276.26$.
\end{enumerate}
Show that if $p_0 > 1 / 2$, you cannot expect to make a profit.

\begin{ex}[P. 401, e. 1]
  Let $X$ be a continuous random variable with values in $[0,2]$ and density $f_X$. Find the moment generating function $g(t)$ for $X$ if \setlength{\parskip}{0cm}
\end{ex}
\begin{enumerate}[label=(\alph*)]
  \item $f_X(x) = \frac{1}{2}$.
  \begin{dmath*}
    g_X(t) = \int_{0}^{2} e^{tx} \cdot \frac{1}{2} dx
           = {\int_{0}^{2} \frac{e^{tx}}{2} dx ; u = tx \rightarrow \frac{du}{dx} = t \rightarrow dx = \frac{du}{t}}
           = \frac{1}{2t} \int_{0}^{2} e^u du
           = {\frac{1}{2t} \left. e^tx \right| _0^2 = \frac{1}{2t} (e^{2t} - e^0)}
           = \frac{e^{2t} - 1}{2t}.
  \end{dmath*}

  \item $f_X (x) = \frac{1}{2}x$.
  \begin{dmath*}
    g_X(t) = \int_{0}^{2} e^{tx} \cdot \frac{1}{2}x dx
           = \frac{1}{2} \int_{0}^{2} x e^{tx} dx
           = {\frac{1}{2} \left[\frac{x e^{tx}}{t} - \int_0^2 \left( (1) \frac{e^{tx}}{t} \right)\right] dx ; u = tx \rightarrow \frac{du}{dx} = t \rightarrow dx = \frac{du}{t}}
           = \frac{1}{2} \left[\frac{x e^{tx}}{t} - \int_0^2 \frac{e^{u}}{t^2} du\right]
           = \frac{1}{2} \left[\frac{x e^{tx}}{t} - \frac{1}{t^2} \int_0^2 e^{tx} dx\right]
           = \frac{1}{2} \left[\left. \frac{x e^{tx}}{t} - \frac{e^{tx}}{t^2} \right|_0^2\right]
           = \frac{1}{2} \left[\left. \frac{tx e^{tx} - e^{tx}}{t^2} \right|_0^2\right]
           = \frac{1}{2} \left[\left. \frac{e^{tx}(tx - 1)}{t^2} \right|_0^2\right]
           = \frac{1}{2} \left[ \frac{e^{2t}(2t - 1)}{t^2} - \frac{(1)(- 1)}{t^2} \right]
           = \frac{e^{2t}(2t - 1) + 1}{2t^2}.
  \end{dmath*}
  \item $f_X (x) = 1 - \frac{1}{2}x$.
  \begin{dmath*}
    g_X(t) = \int_{0}^{2} e^{tx} \cdot (1 - \frac{1}{2}x) dx
           = -\frac{1}{2} \int_0^2 (x  - 2) e^{tx} dx
           = {-\frac{1}{2} \left[ (x - 2) \frac{e^{tx}}{t} - \int_0^2 (1) \frac{e^{tx}}{t} dx \right] ; u = tx \rightarrow \frac{du}{dx} = t \rightarrow dx = \frac{du}{t}}
           = -\frac{1}{2} \left[ \frac{(x - 2) e^{tx}}{t} - \frac{1}{t^2} \int_0^2 e^{u} du \right]
           = -\frac{1}{2} \left[ \left. \frac{(x - 2) e^{tx}}{t} - \frac{e^{tx}}{t^2} \right|_0^2 \right]
           = -\frac{1}{2} \left[ \left. \frac{(x - 2) t e^{tx} - e^{tx}}{t^2} \right|_0^2 \right]
           = -\frac{1}{2} \left[ \left. \frac{e^{tx}[t(x - 2) - 1]}{t^2} \right|_0^2 \right]
           = -\frac{1}{2} \left[ \left( \frac{e^{2t}[t(2 - 2) - 1]}{t^2} \right) - \left( \frac{e^{0t}[t(0 - 2) - 1]}{t^2} \right) \right]
           = -\frac{1}{2} \left[ \left( \frac{-e^{2t}}{t^2} \right) - \left( \frac{-2t - 1}{t^2} \right) \right]
           = -\frac{1}{2} \left[ \frac{-e^{2t} + 2t + 1}{t^2} \right]
           = \frac{e^{2t} - 2t - 1}{2t^2}.
  \end{dmath*}

  \item $f_X (x) = |1 - x|$.
  \begin{dmath*}
    g_X(t) = \int_{0}^{2} e^{tx} |1 - x| dx
           = \int _0^1\left(1-x\right)e^{tx}dx+\int _1^2\left(-1+x\right)e^{tx}dx
           = \frac{e^t-1}{t}-\frac{e^t}{t}+\frac{e^t}{t^2}-\frac{1}{t^2}-\frac{e^{2t}-e^t}{t}+\frac{2e^{2t}}{t}-\frac{e^{2t}}{t^2}-\frac{e^t}{t}+\frac{e^t}{t^2}
           = \frac{t(e^t-1) - t(e^t) + e^t - 1 - t(e^{2t}-e^t) + t(2e^{2t}) - e^{2t} - t(e^t) + e^t}{t^2}
           = \frac{te^t - t - te^t + e^t - 1 - te^{2t} + te^t + 2te^{2t} - e^{2t} - te^t + e^t}{t^2}
           = \frac{-t + 2e^t - 1 + te^{2t} - e^{2t}}{t^2}
           = \frac{2e^t - t - 1 + e^{2t}(t - 1)}{t^2}.
  \end{dmath*}
  \item $f_X (x) = \frac{3}{8}x^2$.
  \begin{dmath*}
    g_X(t) = \int_{0}^{2} e^{tx} (\frac{3}{8}x^2) dx
           = \frac{3}{8}\cdot \int _0^2e^{tx}x^2dx
           = \frac{3}{8}\left[\frac{e^{tx} x^2}{t} - \int \:\frac{2e^{tx} x}{t}dx\right]^2_0
           = \frac{3}{8}\left[\frac{e^{tx} x^2}{t} - \frac{2}{t}\left(\frac{e^{tx} x}{t}-\frac{e^{tx}}{t^2}\right)\right]^2_0
           = \frac{3}{8}\left(\frac{4e^{2t}}{t} - \frac{2}{t}\left(\frac{2e^{2t}}{t}-\frac{e^{2t}}{t^2}\right)-\frac{2}{t^3}\right)
           = \frac{3}{8}\left(\frac{4e^{2t}}{t} - \frac{4e^{2t}}{t^2} + \frac{2e^{2t}}{t^3} - \frac{2}{t^3}\right)
           = \frac{3}{8}\left(\frac{4t^2e^{2t} - 4te^{2t} + 2e^{2t} - 2}{t^3}\right)
           = \frac{3}{8}\left(\frac{2e^{2t} (2t^2 - 2t + 1) - 2}{t^3}\right)
           = \frac{3}{4}\left(\frac{e^{2t} (2t^2 - 2t + 1) - 1}{t^3}\right).
  \end{dmath*}
\end{enumerate}

\begin{ex}[P. 402, e. 6]
  Let $X$ be a continuous random variable whose characteristic function $k_X (\tau)$ is $k_X(\tau) = e^{-|\tau|}$, $-\infty < \tau < \infty$. Show directly that the density $f_X$ of $X$ is
  \begin{equation*}
      f_X(x) = \frac{1}{\pi (1+x^2)}.
  \end{equation*}
\end{ex}

\begin{dmath*}
  f_{X}(x) = \frac{1}{2\pi} \int_{-\infty}^{\infty} e^{-i x \tau } e^{- |\tau|} d\tau
           = \frac{1}{2\pi} \left[ \int _{-\infty}^0 e^{-i x \tau - (-\tau)} d\tau + \int _0^{\infty}e^{-i x \tau - \tau} d\tau \right]
           = \frac{1}{2\pi} \left[ \int _{-\infty}^0 e^{\tau -i x \tau} d\tau + \int _0^{\infty}e^{-i x \tau - \tau} d\tau \right]
           = \frac{1}{2\pi} \left[ \frac{1}{1 - ix} \int _{-\infty}^0 e^{u} du - \frac{1}{ix + 1} \int _0^{\infty}e^{v} dv \right]
           = \frac{1}{2\pi} \left[ \frac{1}{1 - ix} - \frac{1}{ix + 1} \right]
           = \frac{1}{2\pi} \left[ \frac{ix + 1}{i^2x^2 - 1^2} - \frac{1 - ix}{i^2x^2 - 1^2} \right]
           = \frac{1}{2\pi} \left[ \frac{ix + 1 -ix + 1}{1^2 - i^2x^2} \right]
           = \frac{1}{2\pi} \left[ \frac{2}{1 + x^2} \right]
           = \frac{1}{\pi (1+x^2)}.
\end{dmath*}

\begin{ex}[P. 403, e. 10]
  Let $X_1, X_2, \ldots, X_n$ be an indepentent trials process with density
  \begin{equation*}
    f(x) = \frac{1}{2} e^{-|x|}, -\infty < x < + \infty.
  \end{equation*}
  \begin{enumerate}
    \item Find the mean and variance of $f(x)$.
    \item Find the moment generating function for $X_1, S_n, A_n$, and $S_n^*$.
    \item What can you say about the moment generating function of $S_n^*$ as $n \rightarrow \infty$.
    \item What can you say about the moment generating function of $A_n$ as $n \rightarrow \infty$.
  \end{enumerate}
\end{ex}

% \bibliographystyle{mighelnat}
% \bibliography{Biblio}

\end{document}