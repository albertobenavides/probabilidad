\documentclass[10pt, oneside,spanish]{article}   	% use "amsart" instead of "article" for AMSLaTeX format
\usepackage{geometry}                		% See geometry.pdf to learn the layout options. There are lots.
\geometry{a4paper}                   		% ... or a4paper or a5paper or ... 
\usepackage[spanish, es-noindentfirst]{babel}
\selectlanguage{spanish}
\usepackage[utf8]{inputenc}
%\geometry{landscape}                		% Activate for rotated page geometry
%\usepackage[parfill]{parskip}    		% Activate to begin paragraphs with an empty line rather than an indent
\usepackage{graphicx}				% Use pdf, png, jpg, or eps§ with pdflatex; use eps in DVI mode
								% TeX will automatically convert eps --> pdf in pdflatex		
\usepackage{amssymb}
\usepackage{authblk}
%SetFonts

%SetFonts


\title{Retroalimentación a propuestas de compañeros}

\author[ ]{Alberto Benavides}

\affil[ ]{Posgrado en Ingeniería de Sistemas}
\affil[ ]{Facultad de Ingeniería Mecánica y Eléctrica}
\affil[ ]{Universidad Autónoma de Nuevo León}
\renewcommand\Authands{, }
\date{\today}							% Activate to display a given date or no date

\begin{document}
\maketitle

\section{Johana}

Propuesta: \emph{Realizar un diseño de experimentos para evaluar entre un grupo de personas (por medio de una encuesta) la probabilidad estimada que tienen de ingresar a la Universidad considerando factores como la edad, genero, raza, ingresos mensuales, experiencia laboral, años de educación y si están trabajando.}

Retro: En este caso, me parece que habría que definir sobre qué universidad se desea hacer el estudio, pues una primera dificultad que encuentro es que a veces no se puede acceder a los datos que se propone estudiar ya sea porque no se capturan consistentemente o porque no se comparten por motivos de protección de privacidad. En cuanto a las variables que se desean medir, me parece conveniente revisar la relevancia de las presentes y también si existen otros factores que pueden ser considerados, especialmente porque existe mucha bibliografía y aproximaciones en torno a este tema. 
Respuesta: La encuesta se haría a cualquier persona, ya que se quiere estimar la probabilidad de que se ingrese a la Universidad y escogí las variables porque son los factores que he visto que más influyen en la decisión de entrar o no a la Universidad.


\section{Óscar}

Propuesta: \emph{The second proposal is the forecast for manufacturing operation through time series and Auto-Regressive Integrated Moving Average (ARIMA) model. This will help to forecast sales/demand for a period of time.}

Retro: Este tema resulta de especial interés para muchas empresas, pero generalmente el pronóstico a través de estos modelos depende en gran parte de sus autocorrelaciones parciales, por lo que habría que especificar la empresa y el tipo de datos con que se cuenta. Centralmente es importante determinar la frecuencia de captura de datos y la frecuencia con que se desea pronosticar dichos datos. De todas formas, convendría contemplar la posibilidad de toparse con series de tiempo no estacionarias y también proponer estrategias donde se pasen por alguna metodología para volverlas estacionarias o contemplar otros modelos con los que hacer el pronóstico, como el de Holter-Winters.


\section{Palafox}

Propuesta: \emph{The random-walk hypothesis states that a random walk model provides a good explanation of the variation of stock market prices [Godfrey et al., 1964]. In this project, we will explore some of these models, such as the geometric Brownian motion model [Dunbar], and compare it to the performance of real world stocks.}

Retro: Esta idea de aplicar caminatas aleatorias como metodologías para pronóstico de series de tiempo que se consideran irregulares suena bastante interesante en tanto aproximación estocástica para un proceso que también se entiende como estocástico. Me quedan dudas sobre si en esta investigación se abordarán algunos presupuestos de las series de tiempo tales como estacionareidad o relación con ruido blanco, además de otros modelos que usualmente se utilizan para pronóstico de series de tiempo relacionadas con el mercado de valores.

\end{document}  